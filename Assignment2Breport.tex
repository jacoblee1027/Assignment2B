\section{\textbf{Data Science - Electives}}

Before the consideration of what electives will be beneficial when completing the Data Science major, it is of paramount importance to understand what is being studied. Because the business world is becoming more digitalised, online data has increased exponentially for businesses - due to this vast amount of data, compiling and analysing such information has become imperative to the success of the given industry. Thus, the domain of data science constitutes an evaluation of data in order to identify unseen patterns, derivation of critical information for the business, creation of visualisations that can be interpreted and analysed accordingly, and manifestations of predictive models that will assist business decisions (Simplilearn, 2021 and University of Sydney, 2021). 


With this prerequisite knowledge, then, the one potential selection of an elective should be centered around statistical analysis and critical thinking; the major itself is focused on organising relevant data and analysing such figures for the future success of an organisation. Furthermore, another selection of an elective should be focused on the contextual implementation of data science as a domain of application. Compared to the other domains such as computer science and cybersecurity, the field of data science is heavily based on organising and analysing existing data - while the creation of programs dedicated towards such operations are part of the field (similar to computer science or software development), it is focused more heavily on how large data sets can be interpreted to assist business decisions in the future. Thus, it is a field of study that is continuing to appeal to the sustainable growth within companies - according to the Harvard Business Review, professionals who are exploring this large world of data are in major demand (Davenport, Patil, 2012). Needless to say, success in the Data Science major is largely dependent on understanding how technology is evolving, adapting, and being utilised by current organisations. Conclusively, the two chosen electives should develop two areas while studying the major of Data Science: first, the critical thinking and analytical side of Data Science for practical skill development; second, a contextual knowledge of technology and Data Science as a whole so the student can effectively utilise their aforementioned skill-sets in the real-world.


The first elective that will complement the study of Data Science is \textit{QBUS1040: Foundations of Business Analytics} (University of Sydney, 2021). The aforementioned elective is a unit focused on formulating a base for the student studying a Business Analytics major - in addition to teaching linear algebra and calculus for the handling of multiple variables and creating regression models, theory is covered by discussing models in real-life business contexts. Key ideas of optimisation in statistical estimates are covered, as well as basic program utilisation through languages such as Python (University of Sydney, 2021). 


Despite being a commerce elective, there is an immediate connection with the Data Science major - the basic practice of using different analytic programs will form a foundation for the student who is pursuing an Advanced Computing degree. While the coverage of program functionality and implementation is not as detailed as the computing electives, QBUS1040 serves as a sufficient reinforcement - because the aforementioned unit has a specific focus on analytics, the practical programming skills will be geared towards development in the Data Science major. 


Furthermore, the mathematical computation and practice within QBUS1040 provides an amazing opportunity for critical thinking development. An obvious complement to the Data Science major is that mathematics will lead to important knowledge regarding the plotting and visualisation of data. For example, when creating regression models, the student will be ahead in terms of understanding what variables need to be compared, what the data points actually mean, and how to effectively create graphical models of the said data values. Furthermore, practice in the subject of mathematics will improve the student’s critical thinking abilities - a trait important in the field of Data Science. In mathematics, proven rules and theorems are applied in more complex, extensive problems to derive a solution (Cheng, 2015). Consequently, critical thinking is essential when developing mathematical skills as the obtained knowledge of functions, operations, etc. can be applied and utilised to derive a final solution, regardless of the complexity of the problem. Thus, a quintessential connection between the domain of Data Science and critical thinking lies in the definitions between the two terms. Critical thinking is defined as reflective and reasonable thinking that is independently derived based on given information (Ennis, 1985). While the term itself is very broad and has a plethora of implementations, the focal point of critical thinking we are examining is the application of available information to ultimately analyse and derive a solution. Because the Data Science major is based on understanding and deriving key information from a database of numerous figures, an immediate connection can be made: if the student’s critical thinking skills are highly developed, the individual can identify methods of organising the said database into a compact, efficient manner. Naturally, then, development of critical thinking will lead to success when studying the Data Science major.


In addition to its foundational knowledge, QBUS1040 is a unit that will inevitably lead to a second-year elective named QBUS2810: a business analytics elective that continues from QBUS1040, facilitating the student’s knowledge in statistical analysis and delivering new statistical model creation methods (University of Sydney, 2021). Evidently, QBUS1040 is an elective unit that provides a strong grounding for a given student studying in the Data Science major, as their critical thinking and programming skills are established for future success. However, another key capability that the aforementioned elective provides is its continuity throughout the student’s university studies - in addition to QBUS2810, the student can continue to become more knowledgeable in statistical methods, measures, and practical application practices while taking the required computing units. 


To facilitate knowledge on technological trends and digital media, another elective that will complement the Data Science major is \textit{ARIN1001: The Past and Futures of Digital Cultures} (University of Sydney, 2021). Under this elective, the focus is mainly on digital culture: how technology is shaping society and interactions between individuals (gdsgroup, 2021), while also investigating the history, platforms, and ideas that shape technological trends. As mentioned before, the Data Science major is the study of how programs can be utilised to organise data - by understanding the bigger idea of technology, then, the student can place Data Science in a real-world context. While it may seem rudimentary, contextual knowledge is critical in constructing meaning based on individual reflection and experiences, while also applying one’s acquired skill-sets and knowledge (jbcnschool, 2019). For example, the student studying Data Science may memorise all the statistical calculation methods and programming syntax, but find trouble when placed in an actual organisation where real-life scenarios are constantly evolving. Naturally, understanding the contextual knowledge will also provide meaning to studying the major as a whole - it accentuates why the major is imperative in business success, what preparations may be needed to succeed in the near future, and why specific key operations are typically conducted within Data Science as a domain of application.


In ARIN1001, perhaps the most appealing learning outcome is within the technological trends that can be examined. Within the Data Science field, programs such as Tableau Software, Python, and Excel are continually appealing their benefits to businesses to allow for more efficient, effective data organisation. Additionally, as the business world is transitioning into a digital age where almost all transactions and business operations are conducted online, the success of a business is largely based on online consumer trends (business, 2021). Successful firms, then, will capitalise on key information that are relevant for technological success through methods such as effective online marketing, market affiliates, etc. Thus, the benefit of taking ARIN1001 is two-fold: not only will the student be more knowledgeable about digital advancements, but they will also be aware of what specific data and trends can be analysed and taken advantage of in the real-world. 


ARIN1001 can also be pursued throughout the student’s university career as an elective stream. For example, ARIN1001 can extend to ARIN3610: a third-year unit focused on how technology can influence social class, culture, identity, and digital media as a whole (University of Sydney, 2021). Needless to say, if the student decides to pursue this elective stream, the individual will be highly knowledgeable about the direction of technology in the future, its influence in the real-world context, and how they can utilise such ideas in future data analysis and/or business growth. 


In regards to how the aforementioned electives will assist in a student’s professional career, ARIN1001 and QBUS1040 are selected due to the strong foundational base they create. Assuming that the student studying the Data Science major is in their first year, the success of the student’s performance will be largely dependent on how they can apply core theories in practical scenarios. Consequently, since electives are taken in tandem with the main computing units such as INFO1110, the additional units should serve as reinforced knowledge in the Advanced Computing degree they are pursuing. QBUS1040 provides critical thinking development and basic data analysis tool training, which will complement the introductory computing units - in addition to basic syntax and program functionality, the student will have added knowledge on how they can apply such learning outcomes in the context of the Data Science Major. Furthermore, by studying ARIN1001 the student will understand the underpinning concepts of technology; thus, future studies are placed in context and applied in a meaningful way. Ultimately, while the electives may not have an immediate impact on the student’s professional career, the long-term growth that the student can achieve by taking the QBUS1040 and ARIN1001 will be exponentially higher than other Data Science students who are not genuinely knowledgeable in these areas. 



\section{\textbf{Information Systems - Electives}}

\textbf{Elective 1: ENGG1850 - Introduction to Project Management, (Faculty of Engineering, Project Management Major)}
https://www.sydney.edu.au/courses/units-of-study/2021/engg/engg1850.html


ENGG1850 - Introduction to Project Management is a Table S Elective offered by the School of Project Management under the Faculty of Engineering. Engineering and Project Management can both be linked with Information Systems. Information Systems are often highly technical in nature, operating with complex and modern machinery designed by Electrical and Information Engineers. The running of Information Systems often requires a mix of both business & technical (engineering) backgrounds, while the “interdisciplinary approach” of Systems Engineering \texit{“provides the means to enable the realisation of successful complex systems.”} (SESA, 2021). The Systems Engineering Society of Australia describes both Telecommunications and Information and Communications Technology (ICT) as sectors in which Systems Engineering is important to, reflecting the importance of an engineering background to those entering the field of Information Systems.

While under the Faculty of Engineering, this chosen elective comes from the School of Project Management, a critically important spot of learning for Information Systems, which as an industry covers the design, implementation and management of new projects within a system. More than just a technical background, Information Systems require managerial skills among other interpersonal skills, learning from the School of Project Management would be highly appropriate for an Information Systems graduate.


Specifically, ENGG1850 - Introduction to Project Management is a first year introductory unit from the School of Project Management. Outlined in its unit description as providing an \textit{“overview of project management and its relationship to program and portfolio management and the broader business context. The Unit introduces students to variations in project management as interpreted and applied in different industries. It will cover the nature of the project management profession, project career paths and the graduate qualities sought by employers. It introduces the primary professional standards and project management terminology.”} (USYD, 2021). 

The information taught in this unit can be highly applicable to Information Systems students. Students will learn the ins-and-outs of the development of projects, being able to apply it to information systems in ways such as looking at how workers in a project interact throughout the system development life-cycle, workplace productivity tool TextExpander quotes, \textit{“Developers often think that only one of the seven stages of the system development life cycle applies to them. But, to work at their best, everyone in a software development team should have a good working knowledge of all stages of the SDLC.”} (TextExpander, 2021) Students will also gain an understanding on the “business” side of information systems, looking at concepts such as constraints, feasibility, resources. Something which is often not focused on by information systems workers in technical positions. Students studying this subject also have an opportunity to learn technical skills and tools such as the use of Gantt Charts and other project management tools. Making this unit appropriate in providing students with a holistic set of skills that would benefit them in the workforce.


Understanding how projects in Information Systems works is critically important, looking at the jobs that UNSW outline as the career path for their Information Systems graduates, one can see developer, analyst, consultant and manager being the only titles provided. (UNSW, 2021). All of these professional roles work on information systems projects, working through the stages of software design life cycle, analysing business problems and designing solutions and projects to implement these solutions. Students with project management backgrounds will be more involved in their team projects, more attentive & can be seen as more natural leaders in their positions, boosting chances of success in their respective employment. The benefits of a combination of Information Systems and Project Management can be seen with other educational institutions offering degrees such as “Masters of Information Systems Management” (MQU, 2021) reflecting the desire for information systems graduates who can help manage the information systems projects they work on. The ability to work within projects is also highly important for a students academic success throughout university within information systems and not.



Studying this elective on Project Management can also provide opportunities to students by aiding them gain internships and employment during university. Many companies are looking past grades and experience but also looking for students who apply themselves on personal projects in their domains. A student with project management knowledge can help keep a struggling team project afloat, or can help outline and plan out high quality projects that will make an impression on employers. Only furthering the idea that studying this elective of ENGG1850 - Introduction to Project Management as highly appropriate and beneficial for students studying the Bachelor of Advanced Computing, Majoring in Information Systems.



\textbf{Elective 2: QBUS2810 - Statistical Modelling for Business, (Faculty of Business, Business Analytics Major)}

The University of Sydney offers majors in Information Systems, under the faculty of Engineering, as well as Business Information Systems, under the Business School. Reflecting the nature of Information Systems as an industry, a link between technology and the businesses that utilise it. Information Systems are critical for modern businesses to operate, and businesses provide information systems a purpose and functionality, Harvard Business Review even penned an article in 1976 outlining how \textit{“Advances in computer-based information technology in recent years have led to a wide variety of systems that managers are now using to make and implement decisions.”} (HBR, 1976)  Since then information systems and information technology have only gotten more complex and impressive in their work, and business reliance on information systems has only grown. With the business world requiring employees with minor technical skills utilising information technology, making students studying information systems critical in running modern businesses and the information systems they rely on.  Business analytics as a major is tied with information systems, the data interpretation & analysis systems, decision support systems and other analysis tools business analysts use are fundamentally run with information systems, making a student who has a knowledge on business analytics and the tools used for it a highly skilled graduate.

QBUS2810 - Statistical Modelling for Business is a 2000 Level Business unit, following on from QBUS1040 - Foundations of Business Analytics. An excerpt from the Unit Description states \texit{“Statistical analysis of quantitative data is a fundamental aspect of modern business. The pervasiveness of information technology in all aspects of business means that managers are able to use very large and rich data sets. This unit covers a range of methods to model and analyse the relationships in such data, ... The methods are useful for detecting, analysing and making inferences about patterns and relationships within the data so as to support business decisions.”} {USYD, 2021} Information Systems are critical to the running of Statistical Modelling, often these information systems are used to collect, analyse and process the data for business analysts to make decisions from, so studying this elective unit allows for students to apply their information systems knowledge in a setting more applicable to the business workplace, a significant part of the information systems industry. 

In this unit, students will gain knowledge on statistical modelling and information analysis for businesses, from the QBUS2810 description \texit{“This unit offers an insight into the main statistical methodologies for modelling the relationships in both discrete and continuous business data. This provides the information requirements for a range of specific tasks that are required, e.g. in financial asset valuation and risk measurement, market research, demand and sales forecasting and financial analysis, among others.”} (USYD, 2021) Ultimately this allows for Information Systems majors to learn how the information systems they study and interact with actually impact businesses, & how they are used by the customers.

Studying this unit from the Business Analysis major can give Information Systems majors new information related to the businesses they will utilise their information systems skills with, The unit offers real world skills and experience in a business content, highlighting in the description as, \texit{“The unit emphasises real empirical applications in business, finance, accounting and marketing, using modern software tools.”} (USYD, 2021) .


Learning from this unit, giving a student a business analysis background can provide many opportunities to said student, the business background  can be a pathway to working in information systems in many different fields, most large corporations contain business analysts in their information systems department and having this background can provide students with internship and graduate opportunities due to their technical background as well as new business knowledge. Graduates with both a Computing and Business background are also more sought after in the business field compared to just Commerce students due to the mix of both business soft skills and technical computing skills. Making the choice of studying a business unit such as QBUS2810 very appropriate to help improve students studying Information Systems become much more employable in a wide variety of industries and workplaces, this unit gives real world skills to help students utilise their information technology skills in the business world.
